% Options for packages loaded elsewhere
\PassOptionsToPackage{unicode}{hyperref}
\PassOptionsToPackage{hyphens}{url}
%
\documentclass[
]{article}
\usepackage{lmodern}
\usepackage{amssymb,amsmath}
\usepackage{ifxetex,ifluatex}
\ifnum 0\ifxetex 1\fi\ifluatex 1\fi=0 % if pdftex
  \usepackage[T1]{fontenc}
  \usepackage[utf8]{inputenc}
  \usepackage{textcomp} % provide euro and other symbols
\else % if luatex or xetex
  \usepackage{unicode-math}
  \defaultfontfeatures{Scale=MatchLowercase}
  \defaultfontfeatures[\rmfamily]{Ligatures=TeX,Scale=1}
\fi
% Use upquote if available, for straight quotes in verbatim environments
\IfFileExists{upquote.sty}{\usepackage{upquote}}{}
\IfFileExists{microtype.sty}{% use microtype if available
  \usepackage[]{microtype}
  \UseMicrotypeSet[protrusion]{basicmath} % disable protrusion for tt fonts
}{}
\makeatletter
\@ifundefined{KOMAClassName}{% if non-KOMA class
  \IfFileExists{parskip.sty}{%
    \usepackage{parskip}
  }{% else
    \setlength{\parindent}{0pt}
    \setlength{\parskip}{6pt plus 2pt minus 1pt}}
}{% if KOMA class
  \KOMAoptions{parskip=half}}
\makeatother
\usepackage{xcolor}
\IfFileExists{xurl.sty}{\usepackage{xurl}}{} % add URL line breaks if available
\IfFileExists{bookmark.sty}{\usepackage{bookmark}}{\usepackage{hyperref}}
\hypersetup{
  pdftitle={BIOL 5380},
  hidelinks,
  pdfcreator={LaTeX via pandoc}}
\urlstyle{same} % disable monospaced font for URLs
\usepackage[margin=1in]{geometry}
\usepackage{longtable,booktabs}
% Correct order of tables after \paragraph or \subparagraph
\usepackage{etoolbox}
\makeatletter
\patchcmd\longtable{\par}{\if@noskipsec\mbox{}\fi\par}{}{}
\makeatother
% Allow footnotes in longtable head/foot
\IfFileExists{footnotehyper.sty}{\usepackage{footnotehyper}}{\usepackage{footnote}}
\makesavenoteenv{longtable}
\usepackage{graphicx,grffile}
\makeatletter
\def\maxwidth{\ifdim\Gin@nat@width>\linewidth\linewidth\else\Gin@nat@width\fi}
\def\maxheight{\ifdim\Gin@nat@height>\textheight\textheight\else\Gin@nat@height\fi}
\makeatother
% Scale images if necessary, so that they will not overflow the page
% margins by default, and it is still possible to overwrite the defaults
% using explicit options in \includegraphics[width, height, ...]{}
\setkeys{Gin}{width=\maxwidth,height=\maxheight,keepaspectratio}
% Set default figure placement to htbp
\makeatletter
\def\fps@figure{htbp}
\makeatother
\setlength{\emergencystretch}{3em} % prevent overfull lines
\providecommand{\tightlist}{%
  \setlength{\itemsep}{0pt}\setlength{\parskip}{0pt}}
\setcounter{secnumdepth}{-\maxdimen} % remove section numbering

\title{BIOL 5380}
\author{}
\date{\vspace{-2.5em}}

\begin{document}
\maketitle

Welcome to BIOL 5380, Topics in Biomechanics!!!

In BIOL 5380, we'll explore how physical principles constrain or
contribute to biological processes, including movement, feeding, and
transport. By drawing on physics and mechanical engineering, the
educational goal of the course is to understand how organisms swim, fly,
walk, consume resources, and respond to moving fluids as well as how
their size affects the design of mechanical systems. Central to all
these topics will be investigations of how biological materials (e.g.,
wood, muscle, bone, skin, etc.) influence the mechanical behavior of
complex life forms. The course will prepare students for more in-depth
explorations of other related disciplines including biomechanical
engineering, ergonomics, orthopedics, kinesiology, and sports medicine.

A second goal of the course is to engage in the scientific process, that
is:

learn \(\rightarrow\) hypothesize \(\rightarrow\) experiment
\(\rightarrow\) analyze \(\rightarrow\) communicate

Pursuant to this, you'll engage in class projects that involve reading
primary literature, data collection and analysis, and the presentation
of this work to your peers. These projects will require patience,
resilience, and most of all, curiosity (all qualities a scientist---and
especially a biomechanist---will find very handy in their career!).
You'll find that one very important aspect of the course is challenging
you to be proficient with technology (i.e., software, hardware,
instruments, etc.). Biomechanics is a very, \textbf{very} technical
field. Please rest assured, if you summon your curiosity, patience, and
resilience, you'll find this part of the course very rewarding.

\hypertarget{course-deets}{%
\section{Course Deets}\label{course-deets}}

\textbf{Delivery:} Synchronous in-person lectures Monday and Wednesday
at 1:00 in Carney Hall 308. Asynchronous lab projects launched on
Fridays with a synchronous Zoom meeting.

For those who can't make it to class, we'll be
\href{https://bccte.zoom.us/j/9533582156}{Zooming} and
\href{https://bostoncollege.instructure.com/courses/1615104/external_tools/109499}{recording
lectures}.

\textbf{Instructor:} Christopher P. Kenaley (Prof.~K)

\textbf{email:} kenaley {[}ahhhht{]} bc.edu

\textbf{office hours:} By appointment, in person or over Zoom
\href{https://bccte.zoom.us/j/9533582156}{at this link}

\hypertarget{course-materials}{%
\section{Course Materials}\label{course-materials}}

\hypertarget{required-readings-tools-etc.}{%
\subsection{Required Readings, Tools,
etc.}\label{required-readings-tools-etc.}}

\textbf{Recommended Textbook:} Vogel, S. 2013.
\href{https://www.amazon.com/Comparative-Biomechanics-Lifes-Physical-Second/dp/0691155666\#customerReviews}{\emph{Comparative
Biomechanics: Life's Physical World}.} 2nd ed.~Princeton University
Press.

\textbf{Additional readings and lecture files:} Required readings drawn
from the primary literature are available through the class schedule on
the course site. They should be read before the class for which they are
assigned.

\textbf{Technical requirements:} Completing your assignments will
require proficiency with non-commercial software, including some amount
of low-level programming. Early assignments in the course are meant to
bring you up to speed; however, most students in the class will have to
rely on their patience and curiosity to transfer these new skills to
complete assignments. Prof.~Kenaley promises you that scientific
computing with free, but powerful software (e.g.~R, imageJ, etc.) will
be utterly liberating and rewarding!

\hypertarget{github-account}{%
\subsection{GitHub Account}\label{github-account}}

Why a \href{https://github.com/join}{GitHub account}? GitHub is a handy
place for those developing code and offers free hosting of websites
produced and maintained from a desktop (that's how this site was
constructed). Our site is under the ``class" repository (i.e., a space
where code resides) within the organization
\href{https://github.com/bcbiomech}{bcbiomech} on GitHub. To access our
organization and use the bcbiomech discussion page, you'll need an
account. Once you have an account, I'll be able to add you to our
\href{https://github.com/bcbiomech}{bcbiomech} organization and the
class-wide team for discussion purposes.

\hypertarget{canvas-support}{%
\section{Canvas Support}\label{canvas-support}}

You'll find nearly everything you need for BIOL 5380 here at this site;
however, grades will be posted to our
\href{https://bostoncollege.instructure.com/courses/1615104}{class
Canvas site}.

\hypertarget{course-goals-and-objectives}{%
\section{Course Goals and
Objectives}\label{course-goals-and-objectives}}

\hypertarget{learning-goals}{%
\subsection{Learning Goals}\label{learning-goals}}

This course is structured around the use of physical, mathematical, and
engineering concepts to explore a breadth of biological processes that
span all scales of biological organization, from cells to ecosystems.
The learning goals for BIOL 5380 are based upon the
\href{https://live-visionandchange.pantheonsite.io/wp-content/uploads/2011/03/Revised-Vision-and-Change-Final-Report.pdf}{Core
Competencies} that were identified by the American Association for the
Advancement of Science in their
\href{https://live-visionandchange.pantheonsite.io/about-vc-unpacking-a-movement-2018/}{Vision
and Change in Undergraduate Biology Education initiative}. They include
the ability to:

\begin{itemize}
\item
  Apply the process of science
\item
  Use quantitative reasoning
\item
  Use modeling and simulation
\item
  Tap into the interdisciplinary nature of science
\item
  Communicate and collaborate with other disciplines.
\end{itemize}

\hypertarget{learning-objectives}{%
\subsection{Learning Objectives}\label{learning-objectives}}

Within the context of BIOL 5380, these goals can be articulated in the
following specific learning objectives:

\begin{itemize}
\item
  Develop interesting questions related to how structure relates to
  function within the integrated systems of organisms.
\item
  Integrate concepts across disciplines and beyond biology to answer
  questions in biomechanics.
\item
  Leverage quantitative skills---including modeling, simulations, and
  data visualization---to address questions in biomechanics.
\item
  Communicate answers to questions through clear and concise reports and
  presentation.
\end{itemize}

These objectives model the approach taken by a modern biomechanists and
thus represent an experience that will prepare students for future work
in this and related fields (e.g., orthopedics, sports medicine, physical
therapy, bioengineering, etc.).

\hypertarget{course-format}{%
\section{Course Format}\label{course-format}}

\hypertarget{lectures}{%
\subsection{Lectures}\label{lectures}}

Lectures will be a relaxed experience meant to stimulate discussions
around the major themes and concepts related to biomechanics. We have no
exams or quizzes, but please expect a high level of engagement. That is,
lectures will be more than Prof.~Kenaley talking and students listening.
Pauses for brief illustrative exercises and discussions between all of
us will be common and an important part of the lecture experience.
Lectures will be posted as PDF on our \href{schedule.html}{schedule
page}.

\hypertarget{labs}{%
\subsection{Labs}\label{labs}}

As a course-based research experience (i.e., an AE course), BIOL 5380 is
meant to develop and hone research and quantitative skills related to of
biomechanics. The thrust of this course will be to leverage core
concepts from lecture discussions to explore, quantify, and report on
data collected with your own hand and eyes.

To this end, you will participate in 5 asynchronous lab exercises (i.e.,
Mini-projects, ``MPs'') over the course of the semester. For each MP,
you'll be led out with an MP project description, then perform
experiments and collect data. After you've completed the experiments and
analyzed your data, you'll be asked to synthesize and contextualize your
results in a report. These reports---due on Wednesdays---should be short
(\textless 1000 words), contain relevant figures and table, and discuss
the importance of the results. Each MP is worth up to 50 points toward
your final grade.

Additional details for the MPs are available on our
\href{mini_projects.html}{Mini-Projects page}.

Because of the integrative and technical nature of this course, students
will often feel adrift, unsure of how to proceed. This is by design and
a model of how work in the multi-disciplinary field of biomechanics
unfolds. Nonetheless, students may worry about their grade in the course
and how to do well when challenges emerge at every turn. Fear not! For
each assignment in the course (e.g.,project reports) student or team
engagement and effort will be assessed just as much as how well
objective questions have been answered.

A course of this scope, focusing on so much that is technical, would be
administered best in small groups, meeting mostly face-to-face. However,
in the context of a global pandemic that has surged again in our corner
of the world, I can't in good conscience summon us all to meet in
cramped quarters for our lab exercises. Thus, for each lab meeting,
we'll first convene over Zoom and, after an introduction to the MP
topic, students will move on to perform experiments in pre-assigned
teams independently. I will be available for additional discussion and
problem solving over Zoom and face-to-face during arranged office hours.

\hypertarget{final-project}{%
\subsection{Final Project}\label{final-project}}

Toward the end of the course, you'll take on an independent, team-based
project. The goal of this project is the use your quantitative skills
and knowledge of biomechanical systems amassed while executing the MPs
to address a new question of your choosing. The question should be new
within the context of the course and the subjects of your research need
not be human. However, consider the cooperability of the non-human
subject when choosing what to study. You'll address your question in a
longer ($\sim$ 5-page) report that should follow the same format and
requirements of the MP reports, but with expanded content. The report is
worth 100 points toward the final grade.

In addition to a report, you'll be tasked with synthesizing your project
in a short, 10-minute research presentation. This presentation will be
recorded by you and your team using Zoom and uploaded to a directory for
Prof. Kenaley and your peers to view and critique. The Final Project
presentation is worth 50 points toward the final grade. Additional
details for the FP are available on our \href{final_project.html}{Final
Project page}.

\hypertarget{discussion-board}{%
\subsection{Discussion Board}\label{discussion-board}}

Each week on Monday, you will be asked to submit at least one question
and answer another on our class
\href{https://github.com/orgs/bcbiomech/teams/biol-5380}{discussion
board}. This is hosted on GitHub, just like our course site and will
certainly be an important resource as you work through course material.
When you post a question, you should:

\begin{itemize}
\tightlist
\item
  Be polite and concise.
\item
  Ask a question that has not be asked before.
\item
  Post the question with a short descriptive title (e.g.., ``Problems
  with temp sensor wiring" )
\end{itemize}

When responding to a question, you should:

\begin{itemize}
\tightlist
\item
  Be polite and concise
\item
  Provide unique feedback rather than a duplicated answer.
\end{itemize}

Feel free to post reactions (via the smiley-face icon above the text
box), but this won't count as an answer.

A student is welcome to add another answer to a question that has
already been answered as long as it expands upon the previous answers.
I'll be moderating and commenting myself, issuing feedback at the end of
each week.

The discussion activities are worth 50 points toward the final grade. A
complete discussion post includes both a question (2.5 points) and
answer (2.5 points), for a subtotal of 5 points. We have 11 discussions
posts assigned (weeks 2 to 12), however, you're required to submit just
10. That is, we'll drop the lowest of 11 discussion scores and compile
10 of them for the final discussion grade.

\hypertarget{course-assessment}{%
\section{Course Assessment}\label{course-assessment}}

Your grade will be based on five assignment components: Engagement in
class, posting to our discussion board, 5 Mini-Project reports, and one
independent Final Project report and presentation.

\begin{center}\rule{0.5\linewidth}{0.5pt}\end{center}

\hypertarget{grade-breakdown}{%
\subsubsection{Grade Breakdown}\label{grade-breakdown}}

\begin{longtable}[]{@{}llc@{}}
\toprule
\endhead
\begin{minipage}[t]{0.46\columnwidth}\raggedright
\strut
\end{minipage} & \begin{minipage}[t]{0.15\columnwidth}\raggedright
\strut
\end{minipage} & \begin{minipage}[t]{0.31\columnwidth}\centering
\textbf{Points}\strut
\end{minipage}\tabularnewline
\begin{minipage}[t]{0.46\columnwidth}\raggedright
Class participation and engagement\strut
\end{minipage} & \begin{minipage}[t]{0.15\columnwidth}\raggedright
\strut
\end{minipage} & \begin{minipage}[t]{0.31\columnwidth}\centering
50\strut
\end{minipage}\tabularnewline
\begin{minipage}[t]{0.46\columnwidth}\raggedright
Discussion Board\strut
\end{minipage} & \begin{minipage}[t]{0.15\columnwidth}\raggedright
\strut
\end{minipage} & \begin{minipage}[t]{0.31\columnwidth}\centering
10x5\strut
\end{minipage}\tabularnewline
\begin{minipage}[t]{0.46\columnwidth}\raggedright
Mini-Project Reports\strut
\end{minipage} & \begin{minipage}[t]{0.15\columnwidth}\raggedright
\strut
\end{minipage} & \begin{minipage}[t]{0.31\columnwidth}\centering
5x50\strut
\end{minipage}\tabularnewline
\begin{minipage}[t]{0.46\columnwidth}\raggedright
Final Project Report\strut
\end{minipage} & \begin{minipage}[t]{0.15\columnwidth}\raggedright
\strut
\end{minipage} & \begin{minipage}[t]{0.31\columnwidth}\centering
1x100\strut
\end{minipage}\tabularnewline
\begin{minipage}[t]{0.46\columnwidth}\raggedright
Final Project Presentation\strut
\end{minipage} & \begin{minipage}[t]{0.15\columnwidth}\raggedright
\strut
\end{minipage} & \begin{minipage}[t]{0.31\columnwidth}\centering
1x50\strut
\end{minipage}\tabularnewline
\begin{minipage}[t]{0.46\columnwidth}\raggedright
\strut
\end{minipage} & \begin{minipage}[t]{0.15\columnwidth}\raggedright
\textbf{total}\strut
\end{minipage} & \begin{minipage}[t]{0.31\columnwidth}\centering
\textbf{500}\strut
\end{minipage}\tabularnewline
\bottomrule
\end{longtable}

\begin{center}\rule{0.5\linewidth}{0.5pt}\end{center}

The class engagement portion of your grade will be calculated
qualitatively. All 50 points are on the table, as long as you remain an
active participant in class---asking questions, making relevant points,
etc.

Just a note . . . assessment for project reports will incorporate an
evaluation of an individual's or team's effort and engagement in
addition to more objective components (i.e., answers, conclusions,
etc.). To do well, the formula is simple: students and their teams
should remain persistent, ask for guidance, and make clear conclusions
using the skills and tools they learn in the course.

\hypertarget{responsibilities}{%
\section{Responsibilities}\label{responsibilities}}

\hypertarget{expectations-for-students}{%
\subsection{Expectations for Students}\label{expectations-for-students}}

This technical course will require sublime concentration and focus.
Therefore, we must be clear about our expectations of one another.

\begin{itemize}
\tightlist
\item
  You, the student, are responsible for reading and adhering to all
  aspects of the course syllabus as outlined here on this site.
\item
  You are responsible for organizing and completing all readings,
  discussions, exercises, and projects.
\item
  You are responsible for becoming familiar with course-related software
  (R, RStudio, Zoom, etc.).
\item
  You must be aware of your own progress in the course.
\item
  It is up to you to pursue help and guidance when you feel you need it.
\end{itemize}

\hypertarget{expectations-for-the-instructor}{%
\subsection{Expectations for the
Instructor}\label{expectations-for-the-instructor}}

\begin{itemize}
\tightlist
\item
  I will challenge students intellectually to develop data analysis and
  experimental skills.
\item
  I will make expectations clear in the form of this site, syllabus, and
  course announcements made here and over email.
\item
  I will be consistent and quick in grading assignments.
\item
  I will be available to help and discuss topics during scheduled class
  time, arranged office hours, and on our class
  \href{https://github.com/orgs/bcbiomech/teams/biol-5380}{discussion
  board} .
\end{itemize}

\hypertarget{accessibility}{%
\section{Accessibility}\label{accessibility}}

Boston College is committed to providing accommodations to students,
faculty, staff and visitors with disabilities. Specific documentation
from the appropriate office is required for students seeking
accommodation in Woods College courses. Advanced notice and formal
registration with the appropriate office is required to facilitate this
process. There are two separate offices at BC that coordinate services
for students with disabilities:

\begin{itemize}
\tightlist
\item
  \href{http://www.bc.edu/libraries/help/tutoring.html}{The Connors
  Family Learning Center (CFLC)} coordinates services for students with
  LD and ADHD.
\item
  \href{http://www.bc.edu/offices/dos/subsidiary_offices/disabilityservices.html}{The
  Disabilities Services Office (DSO)} coordinates services for all other
  disabilities.
\end{itemize}

Find out more about BC's commitment to accessibility at
www.bc.edu/sites/accessibility.

\hypertarget{academic-integrity}{%
\section{Academic Integrity}\label{academic-integrity}}

Students in this course, as all others at BC, must produce original work
and cite references appropriately. Failure to cite references is
plagiarism. Academic dishonesty includes, but is not necessarily limited
to, plagiarism, fabrication, facilitating academic dishonesty, cheating
on exams or assignments, or submitting the same material or
substantially similar material to meet the requirements of more than one
course without seeking permission of all instructors concerned.
Scholastic misconduct may also involve, but is not necessarily limited
to, acts that violate the rights of other students, such as depriving
another student of course materials or interfering with another
student's work. Please see the
\href{https://www.bc.edu/content/bc-web/academics/sites/university-catalog/policies-procedures.html\#academic_integrity_policies}{Boston
College policy on academic integrity} for more information.

\hypertarget{email-and-office-hours-policy}{%
\section{Email and Office Hours
Policy}\label{email-and-office-hours-policy}}

If you write Professor Kenaley with a question, he'll try to get back to
you within 24 hours. If you don't receive a response, it's likely
because the answer to your question is available here or on our
discussion site. Rather than write Prof.~Kenaley with questions related
to course mechanics and expectations, please post these in our
\href{https://github.com/orgs/bcorgbio/teams/biol-3140}{discussion
board}, It's probably the case that others have the same questions.

Office hours can be arranged by appointment via email and will take
place over Zoom. Office hours may involve more more than student or
team, so please come equipped with poignant questions and expect that
other students have questions as urgent as yours.

~

Cobbled together by Chris Kenaley

{\href{mailto:kenaley@bc.edu}{\nolinkurl{kenaley@bc.edu}}}

~

\end{document}
